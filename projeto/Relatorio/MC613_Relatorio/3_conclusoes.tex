\section{CONCLUSÕES}
Sem dúvidas a maior conclusão tirada pelo grupo, é a diferença existente ao se realizar o desenvolvimento de aplicações de \emph{software} ou em \emph{hardware}. Abandonar a noção de sequencialidade do código, e descobrir outra maneira de sincronizar as diferentes entidades que compõem o projeto. Além disso, outra grande dificuldade encontrada foi com a sintaxe do código em \emph{VHDL}, que apresenta diferenças significativas em comparação a outras linguagens com as quais estamos mais acostumados.

Além dos problemas já citados, o grupo também encontrou grande dificuldade na validação dos módulos do projeto, diferente do que foi trabalhado nos laboratórios, os módulos eram mais abstratos e complexos, o que complicou a realização de simulações mais complexas para testar o projeto. Sendo este, sem sombra de dúvidas, o ponto de maior falha da nossa implementação.

Apesar dos problemas citados, podemos dizer que houve sucesso na implementação proposta. Visto que, o código entregue é capaz de permitir que qualquer usuário se divirta em uma partida de jogo da velha, seja ela contra outra pessoa, ou contra um modo de jogo computadorizado.